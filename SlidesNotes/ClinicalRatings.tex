% Clinical ratings

\lecture{Application: Clinical Ratings}{Clinical}

\begin{frame}
	\frametitle{Clinicians disagree in AandE}
	\begin{itemize}
		\item Patients in \ac{AE} are continually monitored.
		\begin{itemize}
			\item Heart rate
			\item Blood pressure
			\item Temperature
			\item etc
		\end{itemize}
		\item<2-> Based on these hourly observations, clinicians (in a Glasgow hospital) give each patient an ordinal rating
		\begin{itemize}
			\item A (healthy(ish)), B, C, D, E, F (critical)
		\end{itemize}
		\item<3->These ratings are \emph{subjective}
		\begin{itemize}
			\item How do clinicians disagree? (variance? bias?)
		\end{itemize}
		\item<3->More details of this work in \href{http://dx.doi.org/10.1109/JBHI.2013.2252182}{Rogers et al 2013} and \href{http://dx.doi.org/10.1007/s11222-009-9125-z}{Rogers et al 2010}
	\end{itemize}
\end{frame}

\begin{frame}
	\frametitle{Data}
	\begin{itemize}
		\item $c = 1\ldots C$ clinicians.
		\item $p=1\ldots P$ patients.
		\item For patient $p$, we have $T_p$ observations at times $\mathbf{t}_p = [t_{p1},\ldots,t_{pT_p}]^T$.
		\item $y_{\tau c}^p$ is rating at $t_{p\tau}$ ($\{A,B,C,D,E\}$).
	\end{itemize}
\end{frame}

\begin{frame}
	\frametitle{The model}
		\begin{itemize}
		\item Assumptions:
			\begin{itemize}
				\item We assume that there is an unobserved continuous latent health function for each patient
				\item Each clinician observes this and and corrupts it in two ways:
					\begin{itemize}
						\item Adds Gaussian noise
						\item Adds a constant offset (optimist, pessimist)
					\end{itemize}
				\item Corrupted health is then binned to give category
			\end{itemize}
		\item<2-> The model:
		\begin{itemize}
			\item Patient health is modelled as a GP.
			\item Corrupted health is the auxiliary variable ($q$) -- one set of auxiliary variables per clinician
			\item $q$ is binned to produce rating.
			\item Note that $p(q|f)$ has been generalised from standard normal.
		\end{itemize}
	\end{itemize}
\end{frame}

\begin{frame}
	\frametitle{Model}
	\begin{multicols}{2}
		\begin{figure}[tbh]
			\centering\includegraphics[width=0.9\linewidth]{ClinicalCartoon.pdf}
			\centering\caption{\label{fig:clincartoon}Plates diagram}
		\end{figure}
		\newpage
		\begin{eqnarray}
			\nonumber \blf^p &\sim & {\cal N}(\mathbf{0},\mathbf{C}^p)~~\mbox{[health]}\\
			\nonumber \delta_c &\sim & {\cal N}(0,1)~~\mbox{[offset]}\\
			\nonumber \gamma_c &\sim & {\cal G}(\alpha,\beta)~~\mbox{[precision]}\\
			\nonumber q_{c\tau}^p &\sim & {\cal N}(f_\tau^p + \delta_c,\gamma_c^{-1})\\
			\nonumber P(y_{\tau c}^p=k) &=& \delta(b_k < q_{c\tau}^p < b_{k+1})
		\end{eqnarray}
		Previously auxiliary variables were $z_n\sim {\cal N}(f_n,1)$. This model adds clinician-specific offsets and precisions: $q_{c\tau}^p \sim  {\cal N}(f_\tau^p + \delta_c,\gamma_c^{-1})$.
	\end{multicols}
\end{frame}

\begin{frame}
	\frametitle{Example data generation}
	\begin{figure}[tbh]
		\centering\includegraphics<1>[width=0.8\linewidth]{health.pdf}
		\centering\includegraphics<2>[width=0.8\linewidth]{health_corrupted.pdf}
		\centering\includegraphics<3>[width=0.8\linewidth]{health_corrupted_ratings.pdf}
		\centering\caption{\label{fig:health_example}Example of the generative process described by the model for three clinicans.}
	\end{figure}
\end{frame}

\begin{frame}
	\frametitle{Model inference}
	\begin{itemize}
		\item Gibbs sampling is straightforward
		\begin{itemize}
			\item The latent health function for each patient.
			\[
				p(\blf^p|\ldots) = {\cal N}(\blf^p|\boldsymbol\mu_{\blf^p},\boldsymbol\Sigma_{\blf^p})
			\]
			where:
			\[
				\boldsymbol\Sigma_{\blf^p} = \left((\mathbf{C}^p)^{-1} + \sum_c \gamma_c\mathbf{I}\right)^{-1},~~~\boldsymbol\mu_{\blf^p} = \boldsymbol\Sigma_{\blf^p}\sum_c \gamma_c (\mathbf{q}^p_{c\cdot} - \delta_c)
			\]
			\item The auxiliary variables:
			\[
				p(q_{c\tau}^p|y_{c\tau}^p = k,\ldots) \propto \delta(b_k<q_{c\tau}^p<b_{k+1}) {\cal N}(q_{c\tau}^p|f^p_\tau + \delta_c,\gamma_c^{-1})
			\]
		\end{itemize}
	\end{itemize}
\end{frame}

\begin{frame}
	\begin{itemize}
		\item Gibbs sampling continued:
		\begin{itemize}
			\item The offset and precision for each clinician:
			\[
				p(\delta_c|\ldots) = {\cal N}(\delta_c|\mu_c,\sigma_c^2),~~~p(\gamma_c|\ldots) = {\cal G}(a_c,b_c)
			\]
			where:
			\[
				\sigma_c^2 = \left(1 + N_{c}\right)^{-1},~~~\mu_c = \gamma_c\sigma_c^2\sum_p\sum_\tau(q_{c\tau}^p - f_{\tau}^p)
			\]
			and:
			\[
				a_c = \alpha + N_c/2,~~~b_c = \beta+\frac{1}{2}\sum_p\sum_\tau(q_{c\tau}^p - f_{\tau}^p)^2
			\]
			and $N_c$ is the total number of observations for clinician $c$.
		\end{itemize}
	\end{itemize}
\end{frame}

\begin{frame}
	\frametitle{Inference and Interpretation}
	\begin{itemize}
		\item The offset and precision tell us how the clinicians disagree.
		\begin{itemize}
			\item Low precision indicates a clinician who is very unpredictable (w.r.t the rest)
			\item High offset indicates a precision who consistently rates higher or lower than the norm.
		\end{itemize}
		\item<2->Identifiability: offset for one clinician fixed to 0.
		\item<2->Covariance parameter for GP inferred via Metropolis-Hastings (see e.g. \href{http://citeseerx.ist.psu.edu/viewdoc/summary?doi=10.1.1.45.9111}{Rasmussen 2000})
	\end{itemize}
\end{frame}

\begin{frame}
	\frametitle{Results}
	\begin{figure}[tbh]
		\centering\includegraphics[width=0.8\linewidth]{offset.pdf}
		\centering\caption{\label{fig:offset}Marginal offset posteriors}
	\end{figure}
\end{frame}

\begin{frame}
	\frametitle{Results}
	\begin{figure}[tbh]
		\subfigure[Clinicians 2 and 4]{\includegraphics[width=0.45\linewidth]{P2_C24.pdf}}
		\subfigure[Clinicians 3 and 4]{\includegraphics[width=0.45\linewidth]{P2_C34.pdf}}
		\centering\caption{\label{fig:actualratings}Inferred offsets make sense on inspection of the data.}
	\end{figure}
\end{frame}

\begin{frame}
	\frametitle{Results}
	\begin{figure}[tbh]
		\centering\includegraphics[width=0.8\linewidth]{precision_offset_box_17thApril.pdf}
		\centering\caption{\label{fig:clinicalprecision}Marginal precision posteriors. Clinicians 1, 4, and 8 appear to be the least consistent (wrt the majority)}
	\end{figure}
\end{frame}

\begin{frame}
	\frametitle{Results}
	\begin{multicols}{2}
	\includegraphics[width=\linewidth]{patienttrace}
	\newpage
	\begin{itemize}
		\item Posterior health function for one patient
		\item Shaded area shows boundaries of posterior samples
		\item Dashed lines show maximum and minimum clinician ratings
		\begin{itemize}
			\item Note the range: initially patient rated both A and E!
		\end{itemize}
	\end{itemize}
	\end{multicols}
\end{frame}

\begin{frame}
	\frametitle{INSIGHT}
	\begin{itemize}
		\item After the initial annotation, clinicians went through INSIGHT procedure.
		\item The goal was to make ratings more consistent.
		\item If it succeeded, we should see a reduction in offset and increase in precision in the post-INSIGHT data.
		\item Note: only 5 clinicians remained after INSIGHT
	\end{itemize}
\end{frame}

\begin{frame}
	\frametitle{Post-INSIGHT results}
	\begin{figure}[tbh]
		\subfigure[Offsets]{\includegraphics[width=.45\linewidth]{Offset_compare_before_after_17April.pdf}}\hfill
		\subfigure[Precisions]{\includegraphics[width=.45\linewidth]{Precision_before_after_April19th.pdf}}
		\centering\caption{\label{fig:clinoffprec}Offsets and precision before and after INSIGHT. Offsets get closer to 0, whilst precision increase suggesting greater agreement amongsth clinicians.}
	\end{figure}
\end{frame}

\begin{frame}
	\frametitle{Post-INSIGHT results}
	\centering\includegraphics[width=\linewidth]{patienttrace_after}
	\begin{itemize}
		\item Patient health function before and after INSIGHT
		\item Less smooth after INSIGHT
		\item Range of ratings much reduced
	\end{itemize}
\end{frame}

\begin{frame}
	\frametitle{Inferring category boundaries}
	\begin{itemize}
		\item So far, it has been assumed that all categories are the same size (i.e. the elements of $\mathbf{b}$ are equally spaced).
		\item We can also infer these (with fixed end-points and $\delta_c=0$).
		\item Removes uniform prior assumption over categories.
	\begin{figure}[tbh]
		\subfigure[Before INSIGHT]{\includegraphics[width=0.45\linewidth]{start_thresh.pdf}}\hfill
		\subfigure[After INSIGHT]{\includegraphics[width=0.45\linewidth]{end_thresh.pdf}}
		\centering\caption{\label{fig:clinicalcategories}Posterior mean cateogory boundaries.}
	\end{figure}
	\end{itemize}
\end{frame}

\begin{frame}
	\frametitle{Summary and Conclusions}
	\begin{itemize}
		\item Model allows us to:
		\begin{itemize}
			\item learn something about \emph{how} clinicians disagree and how they rate.
			\item assess the effectiveness of the INSIGHT procedure.
		\end{itemize}
		\item<2-> GP prior:
		\begin{itemize}
			\item Flexible
			\item Required no parametric assumptions about health function
			\item Hyper-parameter ($\gamma$) was inferred in the model (could be patient-specific)
		\end{itemize}
		\item<3-> Auxiliary Variable Trick:
		\begin{itemize}
			\item Not restricted to a standard Gaussian centered on the GP variable.
			\item Incorporated offset and precision without causing additional inference challenges.
		\end{itemize}
	\end{itemize}
\end{frame}

\begin{frame}
	\frametitle{Future work}
	\begin{itemize}
		\item Incorporate measured covariates:
		\begin{itemize}
			\item HR, BP, etc
			\item Predictive model? (would need better ground truth)
		\end{itemize}
		\item Patient-specific covariance parameters
		\item Non-stationary covariance parameters
		\begin{itemize}
			\item Long periods of no change followed by short periods of fast change
		\end{itemize}
		\includegraphics[width=0.7\linewidth]{patienttrace_after}
	\end{itemize}

\end{frame}