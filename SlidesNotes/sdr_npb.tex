\providecommand\forslides{1}

 
%%%%%%%%%%%%%%%%%%%%%%%%%%%%%%%%%%%%%%%%%%%%%%%%%%%%%%%%%%%%%%%%%%%%%%%%%%%%%

\ifnum\forslides=1

  \documentclass{beamer}
  \usepackage{etex}
  \reserveinserts{108}
  \usepackage{animate}
  \renewcommand{\cite}[1]{[Ref in notes]}

\else

  \documentclass[11pt,a4paper,twoside]{report}
  %%%%%%%%%%%%%%%%%%%%%%%%%%%%%%%%%%%%%%%%%%%%%%%%%%%%%%%%%%%%%%%%%%%%%%%%%%%%%
  \usepackage{beamerarticle}
  \usepackage{graphicx}
  \setlength{\parindent}{0in}
  %%%%%%%%%%%%%%%%%%%%%%%%%%%%%%%%%%%%%%%%%%%%%%%%%%%%%%%%%%%%%%%%%%%%%%%%%%%%%
  \renewcommand{\lecture}[2]{\chapter{#1}}
  %%%%%%%%%%%%%%%%%%%%%%%%%%%%%%%%%%%%%%%%%%%%%%%%%%%%%%%%%%%%%%%%%%%%%%%%%%%%%
  \newcommand\NormalArticleFrameTitleMode{%
    \renewcommand{\frametitle}[1]{%
      \marginpar{\raggedright ##1 (\insertframenumber)}%
    }%
  }
  \newcommand\ExercisesArticleFrameTitleMode{
    \renewcommand{\frametitle}[1]{%
      ##1%
    }%
  }
  \NormalArticleFrameTitleMode
  %%%%%%%%%%%%%%%%%%%%%%%%%%%%%%%%%%%%%%%%%%%%%%%%%%%%%%%%%%%%%%%%%%%%%%%%%%%%%
  \renewcommand{\familydefault}{\sfdefault}
  
  %%%%%%%%%%%%%%%%%%%%%%%%%%%%%%%%%%%%%%%%%%%%%%%%%%%%%%%%%%%%%%%%%%%%%%%%%%%%%

\fi

\usepackage{subfigure,amsmath}
\usepackage{float}
\usepackage{subfigure}
\usepackage{verbatim}
% %\usepackage[xindy]{glossaries}
% \usepackage{enumitem}
\usepackage{tikz}
\usepackage{verbatim}
\usepackage{xcolor,colortbl}
\usepackage{acronym}

\acrodef{ML}{Machine Learning}
\acrodef{HCI}{Human-Computer Interaction}
\acrodef{GP}{Gaussian Process}
\acrodef{MVG}{Multi-Variate Gaussian}
\newcounter{taskcounter}

\definecolor{highlightcol}{rgb}{1,0.184,0.275}
\newcommand{\focus}[1]{{\textcolor{highlightcol}{\bf #1}}}
\newcommand{\up}[1]{{\textcolor{red}{\bf #1}}}
\newcommand{\down}[1]{{\textcolor{OliveGreen}{\bf #1}}}
\definecolor{yellowhighlighter}{rgb}{1,0.984,0.8}
\definecolor{nicered}{rgb}{1,0,0}
\definecolor{niceorange}{rgb}{1,0.6,0}
\newcommand{\highlightred}[1]{{\textcolor{nicered}{\bf #1}}}
\newcommand{\highlightorange}[1]{{\textcolor{niceorange}{\bf #1}}}
\newcommand{\todo}[1]{{\textcolor{green}{\bf TODO: #1}}}

\newcommand{\bX}{\mathbf{X}}
\newcommand{\bx}{\mathbf{x}}
\newcommand{\bPars}{\boldsymbol\Theta}
\newcommand{\by}{\mathbf{y}}

\graphicspath{{../Code/GP/}}

\newsavebox{\selvestebox}
\newenvironment{task}
  {\stepcounter{taskcounter}
  \newcommand\colboxcolor{F87A17}%
   \begin{lrbox}{\selvestebox}%
   \begin{minipage}[t]{0.15\textwidth}%\dimexpr\columnwidth-2\fboxsep\relax}
   \textbf{TASK [\arabic{taskcounter}]}%
   \end{minipage}%
   \begin{minipage}[t]{0.8\textwidth}%
   }
%   \begin{minipage}[t]{.45\textwidth}
%  Basic Problems of Single-Photon Polymerization:
%  \begin{itemize}
%  \item layer-by-layer type of manufacturing (limits possible geometries)
%  \item suppression through undesired quenching of radicals
%  \item diffraction limits
%  \end{itemize}
% \end{minipage}
  {\end{minipage}\end{lrbox}%
   \begin{center}
   \fcolorbox[HTML]{\colboxcolor}{FFCC99}{\usebox{\selvestebox}}
   \end{center}}

\newcommand{\LectureTitlePage}{%
    % \setcounter{framenumber}{0}
    \global\def\inserttitle{{Lecture \insertlecturenumber: \insertlecture}}
    \global\def\insertshorttitle{{Lecture \insertlecturenumber: \insertlecture}}
    % \global\def\insertdate{\lecturedate}
    % \global\def\insertshortdate{\lecturedate}
  \titlepage
}

\AtBeginLecture{
    \begin{frame}[plain]
      \LectureTitlePage
    \end{frame}
}
\setbeamertemplate{caption}[numbered]

\title{Non-parametric Bayesian Methods in Machine Learning}
\author{Dr. Simon Rogers\\School of Computing Science\\University of Glasgow\\simon.rogers@glasgow.ac.uk\\@sdrogers}

% \includeonlylecture{Intro,CMLintro,prob_stats_R,viz,Clustering,Clustering2}


\begin{document}

\mode<all>

\begin{frame}
	\titlepage
\end{frame}

\mode<all>

%% Outline.tex
\begin{frame}
	\frametitle{Outline}
	\begin{itemize}
		\item {\bf FIX ME AT THE END}
		\item (My) Bayesian philosophy
		\item Gaussian Processes for Regression and Classification
		\begin{itemize}
			\item GP preliminaries
			\item Classification (including semi-supervised)
			\item Regression application 1: clinical (dis)-agreement
			\item Regressopn application 2: typing on touch-screens
		\end{itemize}
		\item Dirichlet Process flavoured Cluster Models
		\begin{itemize}
			\item DP preliminaries
			\item Idenfitying metabolites
			\item (if time) Cluster models for multiple data views
		\end{itemize}
	\end{itemize}
\end{frame}

\begin{frame}
	\frametitle{About me}
	\begin{itemize}
		\item I'm not a statistican by training (don't ask me to prove anything!).
		\item Education:
		\begin{itemize}
			\item Undergraduate Degree: Electrical and Electronic Engineering (Bristol)
			\item PhD: Machine Learning Techniques for Microarray Analysis (Bristol)
		\end{itemize}
		\item Currently:
		\begin{itemize}
			\item Lecturer: Computing Science
			\item Research Interests: Machine Learning and Applied Statistics in Computational Biology and \ac{HCI}
		\end{itemize}
	\end{itemize}
	
\end{frame}

\mode<all>

%% Bayesian intro
\lecture{Bayesian Inference}{bayes}

\begin{frame}
	\frametitle{Bayesian Inference}
	Standard setup:
	\begin{itemize}
		\item We have some data $\bX = \{\bx_1,\ldots,\bx_N\}$
		\item We have a model $p(\bX|\bPars)$
		\item We define a prior $p(\bPars)$
		\visible<2->{
			\item We use Bayes rule (and typically lots of computation) to compute (or estimate) the posterior:
			\[
				p(\bPars|\bX) = \frac{p(\bX|\bPars)p(\bPars)}{p(\bX)}
			\]
		}
	\end{itemize}
\end{frame}

\begin{frame}
	\frametitle{Why Be Bayesian?}
	\begin{itemize}
		\item<2->Ability to incoroporate prior information?
		\item<3->Ability to compute posterior densities (combine prior with likelihood)?
		\item<4->Ability to compare models via marginal likelihood?
		\item<5->For me: the ability to integrate out model parameters completely\ldots
	\end{itemize}
\end{frame}


\begin{frame}
	\frametitle{Why be Bayesian?}
	\begin{itemize}
		\item We're often not interested in parameter values
		\item We're normally interested in something that is a function of the parameter values e.g.:
		\begin{itemize}
			\item Within \ac{ML} we are often interested in making predictions (predicing $y_*$ from $\bx_*$).
			\item This will often require values of some parameters $\bPars$
			\item Being Bayesian allows us to \emph{average} over uncertainity in parameters when making predictions:
			\[
				p(y_*|\bx_*,\bX) = \int p(y_*|\bx_*,\bPars)p(\bPars|\bX)~d\bPars
			\]		
		\end{itemize}
		\item This for me, is the biggest Bayesian selling point!
	\end{itemize}
\end{frame}

\mode<all>

\input{GPIntro}

\end{document}