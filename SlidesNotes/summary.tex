% Summary
\lecture{Summary}{summary}

\begin{frame}
	\frametitle{Summary: GPs}
	\begin{itemize}
		\item Flexible method for regression
		\item Auxiliary variable trick allows:
		\begin{itemize}
			\item Binary classification
			\item Multi-class classification
			\item Semi-supervised classification
			\item Ordinal regression
		\end{itemize}
		\item Applications:
		\begin{itemize}
			\item Modelling uncertainty in text entry
			\item Investigating the disagreement between \ac{AE} clinicians
		\end{itemize}
	\end{itemize}
\end{frame}


\begin{frame}
	\frametitle{Summary: DPs}
	\begin{itemize}
		\item Infinite prior for mixture models
		\item Infer the number of clusters
		\item No magic: The cost is that we have to define what a cluster is quite precisely
		\item Applications:
		\begin{itemize}
			\item Identifying metabolites
			\item Finding patterns across different datasets
		\end{itemize}
	\end{itemize}
\end{frame}

\begin{frame}
	\frametitle{Summary: all}
	\begin{itemize}
		\item \ac{GP}s and \ac{DP}s allow computationally tractable Bayesian inference
		\item In all applications, inference of the curve, or clustering was only an intermediate step:
		\begin{itemize}
			\item \emph{Typing}: propogate uncertainty in key predictions
			\item \emph{Clinicians}: average over curve to infer characteristics of clinicians
			\item \emph{Metabolomics}: average over clustering to annotate peaks
			\item \emph{Multiview}: average over clusterings to extract biological information
		\end{itemize}
		\item Thinking about what the results of regression / clustering will be used for is useful.
		\item \ac{GP} and \ac{DP} provide access to the posterior and therefore the ability to average over it
	\end{itemize}
\end{frame}